\documentclass[11pt]{article}
% \documentclass[DIV13]{scrartcl}

% \usepackage[utf8]{inputenc}
\usepackage[T1]{fontenc}
% \usepackage[ngerman]{babel}
% \usepackage{lmodern}
% \usepackage{graphicx}
% \usepackage{xspace}
% \usepackage[a4paper,lmargin={2cm},rmargin={2cm},tmargin={2.5cm},bmargin = {2.5cm},headheight = {4cm}]{geometry}
% \usepackage{amsmath,amssymb,amstext,amsthm}
% \usepackage[shortlabels]{enumitem}
% \usepackage[headsepline]{scrlayer-scrpage} 
% \usepackage{titling}
% \usepackage{etoolbox
% \input{listings-rust.sty}
\usepackage{tikz}
\usepackage{listings, listings-rust}
\usepackage[scaled=0.85]{beramono}
\usepackage{tikz-cd}

\lstset{language=Rust, style=boxed}

\begin{document}
    \section*{Challenges 1.4}
    \subsection*{Challenge 1.4.1}
    Implement, as best as you can, the identity function in your favorite
    language (or the second favorite, if your favorite language
    happens to be Haskell).\\
    Solution for Rust:
    \begin{lstlisting}{language=Rust}
pub fn my_id<T>(a:T)->T{a}
    \end{lstlisting}

    \subsection*{Challenge 1.4.2}
    Implement the composition function in your favorite language. It
    takes two functions as arguments and returns a function that is
    their composition.\\
    Solution for Rust:
    \begin{lstlisting}{language=Rust}
pub fn compose<'a,F,G,A,B,C>(f:F, g:G) -> Box<(dyn Fn(A)->C + 'a)>
where
F: Fn(A)->B + 'a,
G: Fn(B)->C + 'a, 
{
    return Box::new(move |a| g(f(a)) );
}
    \end{lstlisting}

    \subsection*{Challenge 1.4.3}
    Write a program that tries to test that your composition function
    respects identity.\\
    This isn't possible. With Rice's theorem it follows, that such questions for Programs are undecidable.

    \subsection*{Challenge 1.4.4}
    Is the world-wide web a category in any sense? Are links morphisms?\\
    With links as morphisms the world-wide web isn't a category. This is because wo don't have composition for links. If website A has a link to website B and B a link to website C there isn't nesecery a link from A to C.\\

    \subsection*{Challenge 1.4.5}
    Is Facebook a category, with people as objects and friendships as
    morphisms?\\
    No because we don't have composition. if A and B ara friends and B and C are friends A and C don't have to be friends. 

    \subsection*{Challenge 1.4.6}
    When is a directed graph a category?\\
    If it is closed under reflexivity and transitivity.

    \section*{Challenges 2.7}
    \subsection*{Challenge 2.7.1}
    Define a higher-order function (or a function object) memoize in
    your favorite language. This function takes a pure function f as
    an argument and returns a function that behaves almost exactly
    like f, except that it only calls the original function once for every
    argument, stores the result internally, and subsequently returns
    this stored result every time it’s called with the same argument.
    You can tell the memoized function from the original by watching
    its performance. For instance, try to memoize a function that
    takes a long time to evaluate. You’ll have to wait for the result
    the first time you call it, but on subsequent calls, with the same
    argument, you should get the result immediately. \\
    \begin{lstlisting}
use std::{collections::HashMap, hash::Hash};

struct Memoized<F,A,B>
where
    F: Fn(A)->B,
    A: Clone + Hash + Eq,
    B: Clone
{
    function : F,
    memory: HashMap<A,B>
}

impl<F,A,B> Memoized<F,A,B>
where
    F: Fn(A)->B,
    A: Clone + Hash + Eq,
    B: Clone
{
    fn new(f:F) -> Memoized<F,A,B>{
        Memoized {
            function: f,
            memory: HashMap::new(),
        }
    }

    fn apply(&self, arg:A)->B {
        if self.memory.contains_key(&arg) {
            return self.memory.get(&arg).unwrap().clone();
        } else {
            let result = (self.function)(arg);
            self.memory.insert(arg.clone(), result.clone());
            return result;
        }
    
    }
}
    \end{lstlisting}

    \subsection*{Challenge 2.7.2}
    Try to memoize a function from your standard library that you
    normally use to produce random numbers. Does it work?\\
    This wouldn't work, because a random number generator isn't a pure function. If memoized it would generate the first number random but then it would always return exactly this memoized random number.

    \subsection*{Challenge 2.7.3}
    Most random number generators can be initialized with a seed.
    Implement a function that takes a seed, calls the random number
    generator with that seed, and returns the result. Memoize that
    function. Does it work?\\
    This would work, because now we just memoize the seed for which a fixed random number generator is generated. The calls to the random number generator are not memoized, so they produce a new random number every time the function is called. The only artifact ist, that if we try to get a new random number generator with the same seed, we don't realy get a new one. We get the one that already was instantiated and therefore we don't get the random number sequence from the start but from where it is currently located.

    \subsection*{Challenge 2.7.4}
    Which of these C++ functions are pure? Try to memoize them
    and observe what happens when you call them multiple times:
    memoized and not.\\
    \begin{description}
        \item[(a):] pure
        \item[(b):] impure, because it depends ond what the user types, so it gives a different value depending on user input not on function input
        \item[(c):] pure in the sense of the return behaviour, but it has an effect an is therefore impure in this regard.
        \item[(d):] impure because the static int y variable ist an inner state of the function. If we call it f(1) it returns 1 if we then call it again with the same argument f(1) it returns 2
    \end{description}

    \subsection*{Challenge 2.7.5}
    How many different functions are there from Bool to Bool? Can
    you implement them all?\\
    There are 4 different functions because the function can have two different values for a True input and two different values for a False input. All the function are 
    \begin{itemize}
        \item f1 b = if b then True else True
        \item f2 b = if b then True else False
        \item f3 b = if b then False else True
        \item f4 b = if b then False else False
    \end{itemize}

    \subsection*{Challenge 2.7.6}
    Draw a picture of a category whose only objects are the types
    Void, () (unit), and Bool; with arrows corresponding to all possible
    functions between these types. Label the arrows with the
    names of the functions.\\
    \begin{figure*}
        \centering
        \begin{tikzcd}[column sep=huge, row sep=7em]
            & Void \arrow[ld, swap, "absurd"] \arrow[rd, "absurd"] & \\
            () \arrow[rr, swap, "true", bend right] \arrow[rr, "false"] & & Bool \arrow[ll, swap, "const", bend right] \arrow["{id, true, false, neg}"', loop, distance=2em, in=305, out=235]
        \end{tikzcd}
        \caption{Category for Challenge 2.7.6}
    \end{figure*}


    \section*{Challenges 3.6}
    \subsection*{Challenge 3.6.1}
    Generate a free category from:
    \paragraph*{(a)} A graph with one node and no edges \\
        The Category just consists of the one node as an object, and the added identity arrow. 
    \paragraph*{(b)} graph with one node and one (directed) edge (hint: this
    edge can be composed with itself) \\
        The Category also has just the one node as an object with an added identity arrow. This time though we also add the arrows for arbitrary sequences of composing the one initial edge. So if $e$ is our edge we get an arrow for $e^n$ for $n \in \mathcal{N}$
    \paragraph*{(c)} A graph with two nodes and a single arrow between them \\
        The category just consists of the two nodes as objects with added identity arrows and the one arrow between them. There is no pair of arrows we can compose besides those with identity arrows, so we don't generate any new arrows. 
    \paragraph*{(d)} A graph with a single node and 26 arrows marked with the
    letters of the alphabet: a, b, c … z. \\
        The category consists of the one node as an object and infinitly many arrows labeled with all the possible strings over the alphabet. So for every $w \in \{a,...,z\}^+$ we have an arrow. The Category can be seen as the category of string concatination. The identity arrow then would be the same as a $\epsilon$ arrow

    \subsection*{Challenge 3.6.2}
    What kind of order is this? \\
    \paragraph*{(a)} A set of sets with the inclusion relation: $A$ is included in $B$
    if every element of $A$ is also an element of $B$.\\
    It's a partial order, its reflexive, transitive, and antisymetric. Also not all Sets have to be comparable for example neither $\{a\} \subseteq \{b\}$ nor $\{b\} \subseteq \{a\}$ holds
    \paragraph*{(b)} C++ types with the following subtyping relation: T1 is a subtype
    of T2 if a pointer to T1 can be passed to a function that
    expects a pointer to T2 without triggering a compilation error. \\
    ???

    \subsection*{Challenge 3.6.3} Considering that Bool is a set of two values True and False, show
    that it forms two (set-theoretical) monoids with respect to, respectively,
    operator \&\& (AND) and || (OR). \\
    \&\& and || are associative and they always yield a Bool. For \&\& the neutral element is 1 because $1 \&\& x = x$ and $x \&\& 1 = x$ and for || it is 0 because $0 || x = x$ and $x || 0 = 0$

    \subsection*{Challenge 3.6.4}
    Represent the Bool monoid with the AND operator as a category:
    List the morphisms and their rules of composition.\\
    \begin{figure*}
        \centering
        \begin{tikzcd}[column sep=huge, row sep=7em]
            Bool \arrow[swap, "{true, false}", loop]
        \end{tikzcd}
        \caption{Category for Challenge 3.6.4}
    \end{figure*}
    The id morphism is $true$ and the composition is $f \circ g = f \&\& g$

    \subsection*{Challenge 3.6.5}
    Represent addition modulo 3 as a monoid category. \\
    \begin{figure*}
        \centering
        \begin{tikzcd}[column sep=huge, row sep=7em]
            Bool \arrow[swap, "{0, 1, 2}", loop]
        \end{tikzcd}
        \caption{Category for Challenge 3.6.5}
    \end{figure*}
    The id morphism is 1 and the composition is $f \circ g = (f + g)\:\textsc{mod}\: 3$

    \section*{Challenges 5.8}
    \subsection*{Challenge 5.8.1}
    Show that the terminal object is unique up to unique isomorphism.\\

    

\end{document}