\documentclass[11pt]{article}
% \documentclass[DIV13]{scrartcl}

% \usepackage[utf8]{inputenc}
\usepackage[T1]{fontenc}
% \usepackage[ngerman]{babel}
% \usepackage{lmodern}
% \usepackage{graphicx}
% \usepackage{xspace}
% \usepackage[a4paper,lmargin={2cm},rmargin={2cm},tmargin={2.5cm},bmargin = {2.5cm},headheight = {4cm}]{geometry}
% \usepackage{amsmath,amssymb,amstext,amsthm}
% \usepackage[shortlabels]{enumitem}
% \usepackage[headsepline]{scrlayer-scrpage} 
% \usepackage{titling}
% \usepackage{etoolbox
% \NeedsTeXFormat{LaTeX2e}[1994/06/01]
\ProvidesPackage{listings-rust}[2018/01/23 Custom Package]

\RequirePackage{color}
\RequirePackage{listings}

\lstdefinelanguage{Rust}{%
  sensitive%
, morecomment=[l]{//}%
, morecomment=[s]{/*}{*/}%
, moredelim=[s][{\itshape\color[rgb]{0,0,0.75}}]{\#[}{]}%
, morestring=[b]{"}%
, alsodigit={}%
, alsoother={}%
, alsoletter={!}%
%
%
% [1] reserve keywords
% [2] traits
% [3] primitive types
% [4] type and value constructors
% [5] identifier
%
, morekeywords={break, continue, else, for, if, in, loop, match, return, while}  % control flow keywords
, morekeywords={as, const, let, move, mut, ref, static}  % in the context of variables
, morekeywords={dyn, enum, fn, impl, Self, self, struct, trait, type, union, use, where}  % in the context of declarations
, morekeywords={crate, extern, mod, pub, super}  % in the context of modularisation
, morekeywords={unsafe}  % markers
, morekeywords={abstract, alignof, become, box, do, final, macro, offsetof, override, priv, proc, pure, sizeof, typeof, unsized, virtual, yield}  % reserved identifiers
%
% grep 'pub trait [A-Za-z][A-Za-z0-9]*' -r . | sed 's/^.*pub trait \([A-Za-z][A-Za-z0-9]*\).*/\1/g' | sort -u | tr '\n' ',' | sed 's/^\(.*\),$/{\1}\n/g' | sed 's/,/, /g'
, morekeywords=[2]{Add, AddAssign, Any, AsciiExt, AsInner, AsInnerMut, AsMut, AsRawFd, AsRawHandle, AsRawSocket, AsRef, Binary, BitAnd, BitAndAssign, Bitor, BitOr, BitOrAssign, BitXor, BitXorAssign, Borrow, BorrowMut, Boxed, BoxPlace, BufRead, BuildHasher, CastInto, CharExt, Clone, CoerceUnsized, CommandExt, Copy, Debug, DecodableFloat, Default, Deref, DerefMut, DirBuilderExt, DirEntryExt, Display, Div, DivAssign, DoubleEndedIterator, DoubleEndedSearcher, Drop, EnvKey, Eq, Error, ExactSizeIterator, ExitStatusExt, Extend, FileExt, FileTypeExt, Float, Fn, FnBox, FnMut, FnOnce, Freeze, From, FromInner, FromIterator, FromRawFd, FromRawHandle, FromRawSocket, FromStr, FullOps, FusedIterator, Generator, Hash, Hasher, Index, IndexMut, InPlace, Int, Into, IntoCow, IntoInner, IntoIterator, IntoRawFd, IntoRawHandle, IntoRawSocket, IsMinusOne, IsZero, Iterator, JoinHandleExt, LargeInt, LowerExp, LowerHex, MetadataExt, Mul, MulAssign, Neg, Not, Octal, OpenOptionsExt, Ord, OsStrExt, OsStringExt, Packet, PartialEq, PartialOrd, Pattern, PermissionsExt, Place, Placer, Pointer, Product, Put, RangeArgument, RawFloat, Read, Rem, RemAssign, Seek, Shl, ShlAssign, Shr, ShrAssign, Sized, SliceConcatExt, SliceExt, SliceIndex, Stats, Step, StrExt, Sub, SubAssign, Sum, Sync, TDynBenchFn, Terminal, Termination, ToOwned, ToSocketAddrs, ToString, Try, TryFrom, TryInto, UnicodeStr, Unsize, UpperExp, UpperHex, WideInt, Write}
, morekeywords=[2]{Send}  % additional traits
%
, morekeywords=[3]{bool, char, f32, f64, i8, i16, i32, i64, isize, str, u8, u16, u32, u64, unit, usize, i128, u128}  % primitive types
%
, morekeywords=[4]{Err, false, None, Ok, Some, true}  % prelude value constructors
% grep 'pub \(type\|struct\|enum\) [A-Za-z][A-Za-z0-9]*' -r . | sed 's/^.*pub \(type\|struct\|enum\) \([A-Za-z][A-Za-z0-9]*\).*/\2/g' | sort -u | tr '\n' ',' | sed 's/^\(.*\),$/{\1}\n/g' | sed 's/,/, /g'    
, morekeywords=[3]{AccessError, Adddf3, AddI128, AddoI128, AddoU128, ADDRESS, ADDRESS64, addrinfo, ADDRINFOA, AddrParseError, Addsf3, AddU128, advice, aiocb, Alignment, AllocErr, AnonPipe, Answer, Arc, Args, ArgsInnerDebug, ArgsOs, Argument, Arguments, ArgumentV1, Ashldi3, Ashlti3, Ashrdi3, Ashrti3, AssertParamIsClone, AssertParamIsCopy, AssertParamIsEq, AssertUnwindSafe, AtomicBool, AtomicPtr, Attr, auxtype, auxv, BackPlace, BacktraceContext, Barrier, BarrierWaitResult, Bencher, BenchMode, BenchSamples, BinaryHeap, BinaryHeapPlace, blkcnt, blkcnt64, blksize, BOOL, boolean, BOOLEAN, BoolTrie, BorrowError, BorrowMutError, Bound, Box, bpf, BTreeMap, BTreeSet, Bucket, BucketState, Buf, BufReader, BufWriter, Builder, BuildHasherDefault, BY, BYTE, Bytes, CannotReallocInPlace, cc, Cell, Chain, CHAR, CharIndices, CharPredicateSearcher, Chars, CharSearcher, CharsError, CharSliceSearcher, CharTryFromError, Child, ChildPipes, ChildStderr, ChildStdin, ChildStdio, ChildStdout, Chunks, ChunksMut, ciovec, clock, clockid, Cloned, cmsgcred, cmsghdr, CodePoint, Color, ColorConfig, Command, CommandEnv, Component, Components, CONDITION, condvar, Condvar, CONSOLE, CONTEXT, Count, Cow, cpu, CRITICAL, CStr, CString, CStringArray, Cursor, Cycle, CycleIter, daddr, DebugList, DebugMap, DebugSet, DebugStruct, DebugTuple, Decimal, Decoded, DecodeUtf16, DecodeUtf16Error, DecodeUtf8, DefaultEnvKey, DefaultHasher, dev, device, Difference, Digit32, DIR, DirBuilder, dircookie, dirent, dirent64, DirEntry, Discriminant, DISPATCHER, Display, Divdf3, Divdi3, Divmoddi4, Divmodsi4, Divsf3, Divsi3, Divti3, dl, Dl, Dlmalloc, Dns, DnsAnswer, DnsQuery, dqblk, Drain, DrainFilter, Dtor, Duration, DwarfReader, DWORD, DWORDLONG, DynamicLibrary, Edge, EHAction, EHContext, Elf32, Elf64, Empty, EmptyBucket, EncodeUtf16, EncodeWide, Entry, EntryPlace, Enumerate, Env, epoll, errno, Error, ErrorKind, EscapeDebug, EscapeDefault, EscapeUnicode, event, Event, eventrwflags, eventtype, ExactChunks, ExactChunksMut, EXCEPTION, Excess, ExchangeHeapSingleton, exit, exitcode, ExitStatus, Failure, fd, fdflags, fdsflags, fdstat, ff, fflags, File, FILE, FileAttr, filedelta, FileDesc, FilePermissions, filesize, filestat, FILETIME, filetype, FileType, Filter, FilterMap, Fixdfdi, Fixdfsi, Fixdfti, Fixsfdi, Fixsfsi, Fixsfti, Fixunsdfdi, Fixunsdfsi, Fixunsdfti, Fixunssfdi, Fixunssfsi, Fixunssfti, Flag, FlatMap, Floatdidf, FLOATING, Floatsidf, Floatsisf, Floattidf, Floattisf, Floatundidf, Floatunsidf, Floatunsisf, Floatuntidf, Floatuntisf, flock, ForceResult, FormatSpec, Formatted, Formatter, Fp, FpCategory, fpos, fpos64, fpreg, fpregset, FPUControlWord, Frame, FromBytesWithNulError, FromUtf16Error, FromUtf8Error, FrontPlace, fsblkcnt, fsfilcnt, fsflags, fsid, fstore, fsword, FullBucket, FullBucketMut, FullDecoded, Fuse, GapThenFull, GeneratorState, gid, glob, glob64, GlobalDlmalloc, greg, group, GROUP, Guard, GUID, Handle, HANDLE, Handler, HashMap, HashSet, Heap, HINSTANCE, HMODULE, hostent, HRESULT, id, idtype, if, ifaddrs, IMAGEHLP, Immut, in, in6, Incoming, Infallible, Initializer, ino, ino64, inode, input, InsertResult, Inspect, Instant, int16, int32, int64, int8, integer, IntermediateBox, Internal, Intersection, intmax, IntoInnerError, IntoIter, IntoStringError, intptr, InvalidSequence, iovec, ip, IpAddr, ipc, Ipv4Addr, ipv6, Ipv6Addr, Ipv6MulticastScope, Iter, IterMut, itimerspec, itimerval, jail, JoinHandle, JoinPathsError, KDHELP64, kevent, kevent64, key, Key, Keys, KV, l4, LARGE, lastlog, launchpad, Layout, Lazy, lconv, Leaf, LeafOrInternal, Lines, LinesAny, LineWriter, linger, linkcount, LinkedList, load, locale, LocalKey, LocalKeyState, Location, lock, LockResult, loff, LONG, lookup, lookupflags, LookupHost, LPBOOL, LPBY, LPBYTE, LPCSTR, LPCVOID, LPCWSTR, LPDWORD, LPFILETIME, LPHANDLE, LPOVERLAPPED, LPPROCESS, LPPROGRESS, LPSECURITY, LPSTARTUPINFO, LPSTR, LPVOID, LPWCH, LPWIN32, LPWSADATA, LPWSAPROTOCOL, LPWSTR, Lshrdi3, Lshrti3, lwpid, M128A, mach, major, Map, mcontext, Metadata, Metric, MetricMap, mflags, minor, mmsghdr, Moddi3, mode, Modsi3, Modti3, MonitorMsg, MOUNT, mprot, mq, mqd, msflags, msghdr, msginfo, msglen, msgqnum, msqid, Muldf3, Mulodi4, Mulosi4, Muloti4, Mulsf3, Multi3, Mut, Mutex, MutexGuard, MyCollection, n16, NamePadding, NativeLibBoilerplate, nfds, nl, nlink, NodeRef, NoneError, NonNull, NonZero, nthreads, NulError, OccupiedEntry, off, off64, oflags, Once, OnceState, OpenOptions, Option, Options, OptRes, Ordering, OsStr, OsString, Output, OVERLAPPED, Owned, Packet, PanicInfo, Param, ParseBoolError, ParseCharError, ParseError, ParseFloatError, ParseIntError, ParseResult, Part, passwd, Path, PathBuf, PCONDITION, PCONSOLE, Peekable, PeekMut, Permissions, PhantomData, pid, Pipes, PlaceBack, PlaceFront, PLARGE, PoisonError, pollfd, PopResult, port, Position, Powidf2, Powisf2, Prefix, PrefixComponent, PrintFormat, proc, Process, PROCESS, processentry, protoent, PSRWLOCK, pthread, ptr, ptrdiff, PVECTORED, Queue, radvisory, RandomState, Range, RangeFrom, RangeFull, RangeInclusive, RangeMut, RangeTo, RangeToInclusive, RawBucket, RawFd, RawHandle, RawPthread, RawSocket, RawTable, RawVec, Rc, ReadDir, Receiver, recv, RecvError, RecvTimeoutError, ReentrantMutex, ReentrantMutexGuard, Ref, RefCell, RefMut, REPARSE, Repeat, Result, Rev, Reverse, riflags, rights, rlim, rlim64, rlimit, rlimit64, roflags, Root, RSplit, RSplitMut, RSplitN, RSplitNMut, RUNTIME, rusage, RwLock, RWLock, RwLockReadGuard, RwLockWriteGuard, sa, SafeHash, Scan, sched, scope, sdflags, SearchResult, SearchStep, SECURITY, SeekFrom, segment, Select, SelectionResult, sem, sembuf, send, Sender, SendError, servent, sf, Shared, shmatt, shmid, ShortReader, ShouldPanic, Shutdown, siflags, sigaction, SigAction, sigevent, sighandler, siginfo, Sign, signal, signalfd, SignalToken, sigset, sigval, Sink, SipHasher, SipHasher13, SipHasher24, size, SIZE, Skip, SkipWhile, Slice, SmallBoolTrie, sockaddr, SOCKADDR, sockcred, Socket, SOCKET, SocketAddr, SocketAddrV4, SocketAddrV6, socklen, speed, Splice, Split, SplitMut, SplitN, SplitNMut, SplitPaths, SplitWhitespace, spwd, SRWLOCK, ssize, stack, STACKFRAME64, StartResult, STARTUPINFO, stat, Stat, stat64, statfs, statfs64, StaticKey, statvfs, StatVfs, statvfs64, Stderr, StderrLock, StderrTerminal, Stdin, StdinLock, Stdio, StdioPipes, Stdout, StdoutLock, StdoutTerminal, StepBy, String, StripPrefixError, StrSearcher, subclockflags, Subdf3, SubI128, SuboI128, SuboU128, subrwflags, subscription, Subsf3, SubU128, Summary, suseconds, SYMBOL, SYMBOLIC, SymmetricDifference, SyncSender, sysinfo, System, SystemTime, SystemTimeError, Take, TakeWhile, tcb, tcflag, TcpListener, TcpStream, TempDir, TermInfo, TerminfoTerminal, termios, termios2, TestDesc, TestDescAndFn, TestEvent, TestFn, TestName, TestOpts, TestResult, Thread, threadattr, threadentry, ThreadId, tid, time, time64, timespec, TimeSpec, timestamp, timeval, timeval32, timezone, tm, tms, ToLowercase, ToUppercase, TraitObject, TryFromIntError, TryFromSliceError, TryIter, TryLockError, TryLockResult, TryRecvError, TrySendError, TypeId, U64x2, ucontext, ucred, Udivdi3, Udivmoddi4, Udivmodsi4, Udivmodti4, Udivsi3, Udivti3, UdpSocket, uid, UINT, uint16, uint32, uint64, uint8, uintmax, uintptr, ulflags, ULONG, ULONGLONG, Umoddi3, Umodsi3, Umodti3, UnicodeVersion, Union, Unique, UnixDatagram, UnixListener, UnixStream, Unpacked, UnsafeCell, UNWIND, UpgradeResult, useconds, user, userdata, USHORT, Utf16Encoder, Utf8Error, Utf8Lossy, Utf8LossyChunk, Utf8LossyChunksIter, utimbuf, utmp, utmpx, utsname, uuid, VacantEntry, Values, ValuesMut, VarError, Variables, Vars, VarsOs, Vec, VecDeque, vm, Void, WaitTimeoutResult, WaitToken, wchar, WCHAR, Weak, whence, WIN32, WinConsole, Windows, WindowsEnvKey, winsize, WORD, Wrapping, wrlen, WSADATA, WSAPROTOCOL, WSAPROTOCOLCHAIN, Wtf8, Wtf8Buf, Wtf8CodePoints, xsw, xucred, Zip, zx}
%
, morekeywords=[5]{assert!, assert_eq!, assert_ne!, cfg!, column!, compile_error!, concat!, concat_idents!, debug_assert!, debug_assert_eq!, debug_assert_ne!, env!, eprint!, eprintln!, file!, format!, format_args!, include!, include_bytes!, include_str!, line!, module_path!, option_env!, panic!, print!, println!, select!, stringify!, thread_local!, try!, unimplemented!, unreachable!, vec!, write!, writeln!}  % prelude macros
}%

\lstdefinestyle{colouredRust}%
{ basicstyle=\ttfamily%
, identifierstyle=%
, commentstyle=\color[gray]{0.4}%
, stringstyle=\color[rgb]{0, 0, 0.5}%
, keywordstyle=\bfseries% reserved keywords
, keywordstyle=[2]\color[rgb]{0.75, 0, 0}% traits
, keywordstyle=[3]\color[rgb]{0, 0.5, 0}% primitive types
, keywordstyle=[4]\color[rgb]{0, 0.5, 0}% type and value constructors
, keywordstyle=[5]\color[rgb]{0, 0, 0.75}% macros
, columns=spaceflexible%
, keepspaces=true%
, showspaces=false%
, showtabs=false%
, showstringspaces=true%
}%

\lstdefinestyle{boxed}{
  style=colouredRust%
, numbers=left%
, firstnumber=auto%
, numberblanklines=true%
, frame=trbL%
, numberstyle=\tiny%
, frame=leftline%
, numbersep=7pt%
, framesep=5pt%
, framerule=10pt%
, xleftmargin=15pt%
, backgroundcolor=\color[gray]{0.97}%
, rulecolor=\color[gray]{0.90}%
}

\usepackage{tikz}
\usepackage{listings, listings-rust}
\usepackage[scaled=0.85]{beramono}
\usepackage{tikz-cd}

\lstset{language=Rust, style=boxed}

\begin{document}

\section*{Challenges 1.4}
    \subsection*{Challenge 1.4.1}
        Implement, as best as you can, the identity function in your favorite
        language (or the second favorite, if your favorite language
        happens to be Haskell).\\
        Solution for Rust:

        \begin{lstlisting}{language=Rust}
pub fn my_id<T>(a:T)->T{a}
        \end{lstlisting}

    \subsection*{Challenge 1.4.2}
        Implement the composition function in your favorite language. It
        takes two functions as arguments and returns a function that is
        their composition.\\
        Solution for Rust:

        \begin{lstlisting}{language=Rust}
pub fn compose<'a,F,G,A,B,C>(f:F, g:G) -> Box<(dyn Fn(A)->C + 'a)>
where
F: Fn(A)->B + 'a,
G: Fn(B)->C + 'a, 
{
    return Box::new(move |a| g(f(a)) );
}
        \end{lstlisting}

    \subsection*{Challenge 1.4.3}
        Write a program that tries to test that your composition function respects identity.\\
        This isn't possible. With Rice's theorem ($S = \{ f \in \mathcal{R} | f(id,g) = g, f(g, id) = g\}$) it follows, that such questions for Programs are undecidable. But some Tests could be written as follows.

        \begin{lstlisting}
#[cfg(test)]
mod tests {
    use super::*;
    #[test]
    fn test_compose_id(){
        let foo = compose(id::<i16>, id::<i16>);
        for i in i16::MIN..i16::MAX {
            assert!(i == foo(i));
        }
        let foo = compose(|x:i16| x + 5, id);
        for i in i16::MIN..i16::MAX-5 {
            assert!(i + 5 == foo(i));
        }
        let foo = compose(id, |x:i16| x + 5);
        for i in i16::MIN..i16::MAX-5 {
            assert!(i + 5 == foo(i));
        }
    }
}
        \end{lstlisting}

    \subsection*{Challenge 1.4.4}
        Is the world-wide web a category in any sense? Are links morphisms?\\
        With links as morphisms the world-wide web isn't a category. This is because wo don't have composition for links. If website A has a link to website B and B a link to website C there isn't nesecery a link from A to C. Also not every website has a Link to itself, so also the identity morphism isn't provided.\\
        We can still make it into a category if we don't use links as morphisms but the transitive reflexive closure of those links. like this we get a reachability graph for websites wich is a category. 

    \subsection*{Challenge 1.4.5}
        Is Facebook a category, with people as objects and friendships as
        morphisms?\\
        No because we don't have composition. if A and B ara friends and B and C are friends A and C don't have to be friends. 

    \subsection*{Challenge 1.4.6}
        When is a directed graph a category?\\
        If it is closed under reflexivity and transitivity.

\section*{Challenges 2.7}
    \subsection*{Challenge 2.7.1}
        Define a higher-order function (or a function object) memoize in
        your favorite language. This function takes a pure function f as
        an argument and returns a function that behaves almost exactly
        like f, except that it only calls the original function once for every
        argument, stores the result internally, and subsequently returns
        this stored result every time it’s called with the same argument.
        You can tell the memoized function from the original by watching
        its performance. For instance, try to memoize a function that
        takes a long time to evaluate. You’ll have to wait for the result
        the first time you call it, but on subsequent calls, with the same
        argument, you should get the result immediately. \\
        \begin{lstlisting}
use std::{collections::HashMap, hash::Hash};

struct Memoized<F,A,B>
where
    F: Fn(&A)->B,
    A: Clone + Hash + Eq,
    B: Clone
{
    function : F,
    memory: HashMap<A,B>
}

impl<F,A,B> Memoized<F,A,B>
where
    F: Fn(&A)->B,
    A: Clone + Hash + Eq,
    B: Clone
{
    fn new(f:F) -> Memoized<F,A,B>{
        Memoized {
            function: f,
            memory: HashMap::new(),
        }
    }
    fn apply(&mut self, arg: &A)->B {
        if self.memory.contains_key(&arg) {
            return self.memory.get(&arg).unwrap().clone();
        } else {
            let result = (self.function)(arg);
            self.memory.insert(arg.clone(), result.clone());
            return result;
        }
    }
}
        \end{lstlisting}

    \subsection*{Challenge 2.7.2}
        Try to memoize a function from your standard library that you
        normally use to produce random numbers. Does it work?\\
        This wouldn't work, because a random number generator isn't a pure function. If memoized it would generate the first number random but then it would always return exactly this memoized random number.

    \subsection*{Challenge 2.7.3}
        Most random number generators can be initialized with a seed.
        Implement a function that takes a seed, calls the random number
        generator with that seed, and returns the result. Memoize that
        function. Does it work?\\
        This would work, because now we just memoize the seed for which a fixed random number generator is generated. The calls to the random number generator are not memoized, so they produce a new random number every time the function is called. The only artifact ist, that if we try to get a new random number generator with the same seed, we don't realy get a new one. We get the one that already was instantiated and therefore we don't get the random number sequence from the start but from where it is currently located.

    \subsection*{Challenge 2.7.4}
        Which of these C++ functions are pure? Try to memoize them
        and observe what happens when you call them multiple times:
        memoized and not.\\
        \begin{description}
            \item[(a):] pure
            \item[(b):] impure, because it depends ond what the user types, so it gives a different value depending on user input not on function input
            \item[(c):] pure in the sense of the return behaviour, but it has an effect an is therefore impure in this regard.
            \item[(d):] impure because the static int y variable ist an inner state of the function. If we call it f(1) it returns 1 if we then call it again with the same argument f(1) it returns 2
        \end{description}

    \subsection*{Challenge 2.7.5}
        How many different functions are there from Bool to Bool? Can
        you implement them all?\\
        There are 4 different functions because the function can have two different values for a True input and two different values for a False input. All the function are 
        \begin{itemize}
            \item f1 b = if b then True else True
            \item f2 b = if b then True else False
            \item f3 b = if b then False else True
            \item f4 b = if b then False else False
        \end{itemize}

    \subsection*{Challenge 2.7.6}
        Draw a picture of a category whose only objects are the types
        Void, () (unit), and Bool; with arrows corresponding to all possible
        functions between these types. Label the arrows with the
        names of the functions.\\
        \begin{figure*}
            \centering
            \begin{tikzcd}[column sep=huge, row sep=7em]
                & Void \arrow[ld, swap, "absurd"] \arrow[rd, "absurd"] & \\
                () \arrow[rr, swap, "true", bend right] \arrow[rr, "false"] & & Bool \arrow[ll, swap, "const", bend right] \arrow["{id, true, false, neg}"', loop, distance=2em, in=305, out=235]
            \end{tikzcd}
            \caption{Category for Challenge 2.7.6}
        \end{figure*}


\section*{Challenges 3.6}
    \subsection*{Challenge 3.6.1}
        Generate a free category from:
        \paragraph*{(a)} A graph with one node and no edges \\
            The Category just consists of the one node as an object, and the added identity arrow. 
        \paragraph*{(b)} graph with one node and one (directed) edge (hint: this
        edge can be composed with itself) \\
            The Category also has just the one node as an object with an added identity arrow. This time though we also add the arrows for arbitrary sequences of composing the one initial edge. So if $e$ is our edge we get an arrow for $e^n$ for $n \in \mathcal{N}$
        \paragraph*{(c)} A graph with two nodes and a single arrow between them \\
            The category just consists of the two nodes as objects with added identity arrows and the one arrow between them. There is no pair of arrows we can compose besides those with identity arrows, so we don't generate any new arrows. 
        \paragraph*{(d)} A graph with a single node and 26 arrows marked with the
        letters of the alphabet: a, b, c … z. \\
            The category consists of the one node as an object and infinitly many arrows labeled with all the possible strings over the alphabet. So for every $w \in \{a,...,z\}^+$ we have an arrow. The Category can be seen as the category of string concatination. The identity arrow then would be the same as a $\epsilon$ arrow

    \subsection*{Challenge 3.6.2}
        What kind of order is this? \\
        \paragraph*{(a)} A set of sets with the inclusion relation: $A$ is included in $B$
        if every element of $A$ is also an element of $B$.\\
        It's a partial order, its reflexive, transitive, and antisymetric. Also not all Sets have to be comparable for example neither $\{a\} \subseteq \{b\}$ nor $\{b\} \subseteq \{a\}$ holds
        \paragraph*{(b)} C++ types with the following subtyping relation: T1 is a subtype
        of T2 if a pointer to T1 can be passed to a function that
        expects a pointer to T2 without triggering a compilation error. \\
        ???

    \subsection*{Challenge 3.6.3}
        Considering that Bool is a set of two values True and False, show
        that it forms two (set-theoretical) monoids with respect to, respectively,
        operator \&\& (AND) and || (OR). \\
        \&\& and || are associative and they always yield a Bool. For \&\& the neutral element is 1 because $1 \&\& x = x$ and $x \&\& 1 = x$ and for || it is 0 because $0 || x = x$ and $x || 0 = 0$

    \subsection*{Challenge 3.6.4}
        Represent the Bool monoid with the AND operator as a category:
        List the morphisms and their rules of composition.\\
        \begin{figure*}
            \centering
            \begin{tikzcd}[column sep=huge, row sep=7em]
                Bool \arrow[swap, "{true, false}", loop]
            \end{tikzcd}
            \caption{Category for Challenge 3.6.4}
        \end{figure*}
        The id morphism is $true$ and the composition is $f \circ g = f \&\& g$

    \subsection*{Challenge 3.6.5}
        Represent addition modulo 3 as a monoid category. \\
        \begin{figure*}
            \centering
            \begin{tikzcd}[column sep=huge, row sep=7em]
                Bool \arrow[swap, "{0, 1, 2}", loop]
            \end{tikzcd}
            \caption{Category for Challenge 3.6.5}
        \end{figure*}
        The id morphism is 1 and the composition is $f \circ g = (f + g)\:\textsc{mod}\: 3$

\section*{Challenges 5.8}
    \subsection*{Challenge 5.8.1}
        Show that the terminal object is unique up to unique isomorphism.\\

    
\section*{Challenges 8.9}

    \subsection*{8.9.1}

        \begin{lstlisting}[language=Haskell]
data Pair a b = Pair a b

instance Bifunctor Pair where
    bimap f g (Pair a b) = Pair (f a) (g b)
    first f (Pair a b) = Pair (f a) b
    second f (Pair a b) = Pair a (f b)
        \end{lstlisting}

    \subsection*{8.9.2}

        \begin{lstlisting}
type Maybe' a = Either  (Const () a) (Identity a)

isom :: Maybe a -> Maybe' a
isom Nothing = Left $ Const ()
isom (Just a) = Right $ Identity a

isom' :: Maybe' a -> Maybe a
isom' (Left (Const ())) = Nothing
isom' (Right (Identity a)) = Just a
        \end{lstlisting}

    \subsection*{8.9.3}

        \begin{lstlisting}
data PreList a b = Nil | Cons a b
instance Bifunctor PreList where
    bimap f g Nil = Nil
    bimap f g (Cons a b) = Cons (f a) (g b)
    first f Nil = Nil
    first f (Cons a b) = Cons (f a) b
    second g Nil = Nil
    second g (Cons a b) = Cons a (g b)
        \end{lstlisting}

    \subsection*{8.9.4}

        \begin{lstlisting}
newtype K2 c a b = K2 c
instance Bifunctor (K2 c) where 
    bimap _ _ (K2 c) = K2 c

newtype Fst a b = Fst a
instance Bifunctor Fst where
    bimap f _ (Fst a) = Fst (f a)

newtype Snd a b = Snd b
instance Bifunctor Snd where
    bimap _ g (Snd b) = Snd $ g b
        \end{lstlisting}

\end{document}